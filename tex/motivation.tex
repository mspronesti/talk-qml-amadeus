% path for images
\graphicspath{{assets/motivation/}}

\section[Motivation]{Motivation}

%%
\begin{frame}{Why going Quantum ?}
	Until now, we’ve relied on supercomputers to solve most problems. These are very large classical computers, often with thousands of classical CPU and GPU cores. However, supercomputers are not very good at solving certain types of problems, which seem easy at first glance. 
	\begin{center}
	    \includegraphics[width=.7\linewidth]{supercomputer}
	\end{center}
\end{frame}

%%
\begin{frame}{Why going Quantum ?}
Imagine you want to seat 10 fussy people at a dinner party, where there is only one optimal seating plan out of all the different possible combinations. How many different combinations would you have to explore to find the optimal?

Can you guess how many \alert{combinations}?
\end{frame}

%% animation
\begin{frame}[<+- | only+>]{Why going Quantum ?}
	\metroset{block=fill}
	
	
	\begin{exampleblock}{For 2 people}
		2 Total combination.
	\end{exampleblock}
	

	\begin{block}{For 5 people}
		120 Total combination.
	\end{block}
	

	\begin{alertblock}{For 10 people}
		Over 3 Million of total combination!!!
	\end{alertblock}	
	
	\begin{itemize}
		\item Supercomputers don't have the working \alert{memory} to hold the myriad combinations of real world problems.
		\item Supercomputers have to analyze each combination one after another, which can take a long \alert{time}.
	\end{itemize}
\end{frame}

\begin{frame}{Why is Quantum faster?}
\begin{figure}[H]
    \centering
    \includegraphics[width=.7\linewidth]{ quantumsup.png}
    \caption{Quantum Supremacy}
\end{figure}
\begin{itemize}
    \item Quantum computers can create vast multidimensional spaces in which to represent these very large problems. Classical supercomputers cannot do this.
   
\end{itemize}
  
\end{frame}

\begin{frame}{Why is Quantum faster?}
\begin{itemize}
 \item Referring to the "10 fussy people at a dinner party" problem, with \alert{22 qubit} we can represent $2^{22} = 4194304$ states.
    \item The computation may be carried out on all those numbers in a \alert{single parallel computation}. This built-in parallelism is the key to the power of quantum computers.
\end{itemize}
    
\end{frame}

%% 
\begin{frame}{Quantum Computer}
%%%%%%%%%%%%%%%%%%%%%%%%%%%
%% Ste due immagini messe così fanno schifo al cazzo
%% devono essere stessa altezza e, sopratutto, quando si mettono due 
%% immagini allineate non si trattano come fossero due figure distinte ...
%%%%%%%%%%%%%%%%%%%%%%%%%%%
\begin{figure}[!htb]
   \begin{minipage}{0.48\textwidth}
     \centering
     \includegraphics[width=.6\textwidth, height=.3\textheight]{quantum-computer}
     \caption{IBM Quantum Computer}\label{Fig:Data1}
   \end{minipage}\hfill
   \begin{minipage}{0.48\textwidth}
     \centering
     \includegraphics[width=.6\textwidth, height=.3\textheight]{LEPbw}
     \caption{Inside Look}\label{Fig:Data2}
   \end{minipage}
\end{figure}
A quantum computer is a physical realisation of a quantum Turing machine supported by quantum mechanical processes which are modelled as physical qubits and abstracted as logical qubits.
\end{frame}

%%
\begin{frame}{How is a Quantum computer programmed?}
\begin{figure}[H]
    \centering
    \includegraphics[width=.4\linewidth]{ Hto8a.png}
    \caption{Quantum Computer Architecture}
\end{figure}
There is an \alert{interface} between quantum mechanical processes and classical computer processes. Through this interface, input data from a a classical computing device can be fed into a quantum circuit.
\end{frame}

%%
\begin{frame}{How is a Quantum computer programmed?}
    \begin{itemize}
        \item \alert{Quantum Circuits} are constructed from Quantum Registers.
        \item \alert{Quantum Register} is a type of circuit construction from logical qubits. 
        \item \alert{Logical Qubits} can create different permutations and combinations of physical qubit manifestations.
    \end{itemize}
\end{frame}

%%
\begin{frame}[fragile]{What are Quantum Computers good at ?}
    \begin{figure}[!htb]
        \subfloat[Linear Algebra]{\includegraphics[width=0.3\linewidth]{linalg}} \quad
		 \subfloat[Sampling]{\includegraphics[width=0.3\linewidth]{sampling}} \quad
		 \subfloat[Optimization]{\includegraphics[width=0.33\linewidth, height=0.38\textheight]{optimiz}} 
    \end{figure}
    %
	%cose da dire: 
	%\begin{itemize}
	%	\item Linear Algebra
	%	\item Optimization
	%	\item Sampling
	%	\item Research Algorithms
	%\end{itemize}
\end{frame}


%%
\begin{frame}[fragile]{A Growing Interest in the field}
	cose da dire: 
	\begin{itemize}
	 \item tanti framework (qiskit, pennylane, cirq, tensorflow quantum, ...)
	 \item  tanti paperi (con plot della crescita 2012-2021)
	 \item  tutti (Microsoft, Google, IBM, NVidia ... ) hanno/vogliono un computer quantistico
	 \item  tanta ricerca nel settore Quantum Programming Languages
	\end{itemize}
\end{frame}